\documentclass[a4paper, 11pt, danish, final]{article}
\usepackage{bonde}

\def\mytitle{Dataanalyse 2010}
\def\mysubtitle{Aflevering af ugeopgave 3}
\def\myauthor{Ulrik Bonde}
\def\mymail{\mailto{bonde@diku.dk}}
\def\mydate{\today}
\def\repository{\url{http://github.com/bonde/dataanalyse}}

\title{\mytitle}
\subtitle{\mysubtitle}

\author{\myauthor{} - \mymail}
\date{\mydate}

\hypersetup{
colorlinks,%
citecolor=black,%
filecolor=black,%
linkcolor=black,%
urlcolor=black,%
bookmarksopen=false,
pdftitle={\mytitle{} - \mysubtitle},
pdfauthor={\myauthor}
}

\begin{document}
\maketitle

\subsection*{Spørgsmål 1}
\paragraph{Ligefordeling}
Hvis vi antager at vægten af \emph{middelstore} strandsten i
Jammerbugten er ligefordelt, medfører dette at man må forvente at der er
lige mange sæt af sten som vejer det samme, lige mange sæt af sten hvor
den ene er tungere end den anden og vice versa. Vi har altså taget hver
sten og sammenlignet med enhver anden sten på stranden.

Det er derfor også implicit, at der i Jammerbugten, er et uendeligt
antal middelstore strandsten. Denne antagelse er nok heller ikke helt
urimelig, da større strandsten på et eller andet tidspunkt vil blive
middelstore pga.  erosion og andre naturfænomener som jeg ikke har
videre forstand på. Derfor er selve spørgsmålet, et eller andet sted,
ret dumt, da stenenes vægt ikke er konstant. Dog kan vi ved antagelsen
om ligefordeling se bort fra dette faktum, men vi siger da heller ikke
andet, end at ``der er ret mange middelstore strandsten i
Jammerbugten''.

\paragraph{Empirisk}
Ved at beskrive de middelstore strandsten i Jammerbugten ved empiri
tages et øjebliksbillede af stenene på stranden med en detajlegrad efter
hvor mange prøver der tages.

Vi kan af åbenlyse årsager ikke tage alle sten og sammenligne med enhver
anden. Vi skal derfor overveje, om vi i vores stikprøver vil ligge sten
tilbage hvis vi har brugt den i en sammenligning. Endvidere skal vi
beslutte, om vi ved ligevægt skal have, at to sten vejer \emph{præcis}
det samme eller om vi har en margin hvori vi godtager resultater.

Der ligger dog stadig et stort arbejde i den empiriske fremgangsmåde, da
hver sten skal vejes.

\paragraph{Gæt}
Ved gæt skal man overveje hvornår man gætter og på hvilket grundlag.
Hvis der gættes uden overhovedet at have været i Jammerbugten og set de
to pågældende sten, er det svært at stole på.

Gættet kan også foretages blot ved at kigge på stenene uden at samle
dem op. Her kan man da ikke længere være sikker på at stenen er
middelstor, da dele af stenen kan være gemt i sandet.

Endeligt kan man samle stenene op og vurdere vægten. Dette virker
umiddelbart som det mest troværdige resultat ved brug af gæt.

\paragraph{Statistisk uafhængighed}
Jeg mener ikke at det er fornuftigt at antage statistisk uafhængighed,
da vi kun betragter \emph{middelstore} sten. Vi derfor allerførst
overveje hvornår en sten er middelstor. Rokkestenen på Bornholm kan godt
siges at være middelstor i sammenligning med Ayers Rock i Australien. Om
ikke andet sammenlignes kun sten med omtrent samme størrelse, hvilket
gør statistisk uafhængighed besværligt. Vælges en ``større middelstor
sten'' vil man også sammenligne denne med en tilsvarende ``stor
middelstor sten''.

\subsection*{Spørgsmål 2}
Vi skal afgøre hvorvidt det givne kryptosystem er perfekt ved at
evaluere $P(x|y) = P(x)$ for alle $x \in \mathcal{P}$ og $y \in
\mathcal{C}$.

Hvis vi sætter $x = a$ og $y = 1$ skal vi altså have opfyldt ligningen
\begin{align}
    P(a|1) &= P(a)
\end{align}
Ved at studere kryptotabellen, ser vi, at man med $y = 1$ er sikker på
at $x = a$ og vi får da
\begin{align}
    P(a|1) &= \frac{1}{4}\\
    1 &= \frac{1}{4}
\end{align}
hvilket er en modstrid. Kryptosystemet er altså ikke perfekt.
Ovenstående er også gældende for $x = b$ og $y = 4$. Her er man også
sikker på at den krypterede besked har været \texttt{b}.

%%%%%%%%%%%%%%%%%%%%%%%%%%%%%%%%%%%%%%%%%%%%%%%%%%%%%%%%%%%%%%%%%%%%
% Formal stuff

%\bibliographystyle{abbrvnat}
%\bibliography{bibliography}
%\addcontentsline{toc}{chapter}{Litteratur}

\end{document}

% vim: set tw=72 spell spelllang=da:
